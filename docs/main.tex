% ============================================================
%  Hybrid PQ-OPAQUE — Scientific Paper
%  Template: Springer LNCS (llncs)
%  Compiler:  pdfLaTeX (Overleaf default)
% ============================================================
\documentclass{llncs}

% ── Language & Encoding (pdfLaTeX) ───────────────────────────
\usepackage[T2A]{fontenc}
\usepackage[utf8]{inputenc}
\usepackage[english,ukrainian]{babel}

% ── Mathematics ──────────────────────────────────────────────
\usepackage{amsmath}
\usepackage{amssymb}
\usepackage{mathtools}

% ── Tables ───────────────────────────────────────────────────
\usepackage{booktabs}
\usepackage{tabularx}
\usepackage{multirow}
\usepackage{array}
\usepackage{longtable}
\usepackage{makecell}
\usepackage{lscape}

% ── Code Listings ────────────────────────────────────────────
\usepackage{listings}
\usepackage{xcolor}
\lstset{
  basicstyle=\ttfamily\footnotesize,
  breaklines=true,
  frame=single,
  framesep=3pt,
  backgroundcolor=\color{gray!8},
  numbers=none,
  keepspaces=true,
}

% ── Layout & Typography ──────────────────────────────────────
\usepackage{microtype}
\usepackage{caption}
\captionsetup{font=small, labelfont=bf}

% ── Hyperlinks ───────────────────────────────────────────────
\usepackage{hyperref}
\hypersetup{
  colorlinks=true,
  linkcolor=black,
  citecolor=black,
  urlcolor=blue,
}

% ── Custom column types ──────────────────────────────────────
\newcolumntype{L}[1]{>{\raggedright\arraybackslash}p{#1}}
\newcolumntype{C}[1]{>{\centering\arraybackslash}p{#1}}
\newcolumntype{R}[1]{>{\raggedleft\arraybackslash}p{#1}}

% ── Re-define theorem environments in Ukrainian ──────────────
% (llncs already provides theorem, definition, lemma)
\spnewtheorem*{sketchproof}{Доведення (ескіз)}{\itshape}{\rmfamily}

% ============================================================
\begin{document}
% ============================================================

\title{Гібридний постквантовий протокол автентифікації на основі
парольного ключового обміну OPAQUE з інтеграцією ML-KEM-768:
розробка, формальна верифікація та реалізація}

\author{Олександр Мельниченко}

\institute{Ecliptix Security, Україна}

\maketitle

% ============================================================
\begin{abstract}
% ============================================================
Парольна автентифікація залишається домінуючим механізмом підтвердження
особи користувачів у сучасних інформаційних системах. Криптографічну
основу найбільш захищених протоколів парольного автентифікованого
ключового обміну (PAKE), зокрема протоколу OPAQUE, складають операції
на еліптичних кривих та протокол Діффі--Геллмана, безпека яких
ґрунтується на обчислювальній складності задачі дискретного
логарифмування. Поява масштабованих квантових обчислювальних пристроїв,
здатних ефективно реалізувати алгоритм Шора за поліноміальний час,
створює фундаментальну загрозу для зазначених криптографічних
примітивів. Стратегія «збирай зараз~--- дешифруй пізніше» робить цю
загрозу практично значущою вже на теперішній момент.

Метою дослідження є розробка, формальна верифікація та реалізація
гібридного постквантового розширення OPAQUE, що забезпечує одночасну
стійкість як до класичних, так і до квантових криптоаналітичних атак.
Запропоновано конструкцію Гібридного PQ-OPAQUE, в якій механізм
інкапсуляції ключів ML-KEM-768 (FIPS~203) інтегровано в потік
автентифікованого ключового обміну за схемою 4DH як обов'язкову
складову протоколу. Класичний ключовий матеріал від чотирьох операцій
Діффі--Геллмана на кривій Ristretto255 (128~байт) конкатенується зі
спільним секретом ML-KEM-768 (32~байти); об'єднаний вхід обсягом
160~байт опрацьовується функцією HKDF-Extract на основі HMAC-SHA-512
із контекстно-залежною сіллю, що містить мічений геш розширеного
транскрипту.

Безпеку протоколу верифіковано двома незалежними інструментами
символьної верифікації: \textbf{ProVerif~2.05} (5/5 запитів підтверджено)
та \textbf{Tamarin~Prover~1.10.0} (8/8 лем доведено за 28,08~с). Реалізацію
виконано мовою \textbf{Rust~1.93.1} у вигляді чотирикрейтного простору
проєктів загальним обсягом 2\,729~рядків виробничого коду та
2\,700~рядків тестів. Набір тестів включає \textbf{111~тестових сценаріїв}
(0~відмов), що охоплюють сім властивостей безпеки. Вимірювання
продуктивності на платформі Apple~M1~Pro засвідчили: повний цикл
автентифікації (наскрізний)~--- 481,43~мс; ML-KEM-768 (повний
раунд)~--- 122,01~мкс; 4DH ($\approx 164$~мкс); \texttt{generate\_ke2}~---
297,03~мкс. Домінування Argon2id (510,86~мс, 99,8\%) підтверджує
відповідність дизайну вимогам стійкості до атак офлайн-перебирання.
\end{abstract}

\keywords{OPAQUE, PAKE, ML-KEM-768, постквантова криптографія,
гібридний ключовий обмін, Ristretto255, HKDF, Tamarin, ProVerif, Rust}

% ============================================================
\section{Вступ}
% ============================================================

\subsection{Актуальність проблеми}

Парольна автентифікація залишається найбільш поширеним механізмом
підтвердження особи в сучасних розподілених інформаційних системах.
За різними оцінками, понад 80\% процедур автентифікації у веб-додатках
та мережевих сервісах базуються на паролях~\cite{bonneau2012,florencio2007}.
Протоколи парольного автентифікованого ключового обміну (PAKE)
забезпечують встановлення захищеного каналу між двома сторонами, що
поділяють спільний секрет низької ентропії~--- пароль~--- без розкриття
цього паролю противнику~\cite{bellare2000}. Серед відомих протоколів
PAKE протокол OPAQUE~\cite{jarecki2018}, запропонований Jarecki,
Krawczyk та Xu у 2018~році, вважається найбільш безпечним представником
класу асиметричних (augmented) PAKE: сервер зберігає лише
криптографічний запис, з якого неможливо відновити пароль навіть при
повній компрометації серверного сховища~\cite{ietf-opaque}.

\subsection{Квантова загроза}

У 1994~році Пітер Шор запропонував квантовий алгоритм~\cite{shor1994},
здатний розв'язувати задачу дискретного логарифмування за
поліноміальний час на квантовому комп'ютері. Цей алгоритм безпосередньо
загрожує всім криптосистемам на основі DLP та ECDLP:~RSA, ECDH, а
також будь-якому PAKE-протоколу, що будується на їх
основі~\cite{bernstein2017}. За оцінками, для зламу
Ristretto255/Curve25519 достатньо криптографічно значущого квантового
комп'ютера з приблизно 2\,330 логічними кубітами~\cite{roetteler2017}.

Особливу небезпеку становить стратегія «збирай зараз~--- дешифруй
пізніше»~\cite{mosca2018}: противник зберігає зашифрований трафік
сьогодні з метою ретроспективного криптоаналізу на квантовому
комп'ютері в майбутньому. Для систем, що опрацьовують довготривалі
конфіденційні дані, ця загроза є практично значущою вже зараз.

\subsection{Стандартизація постквантової криптографії}

У серпні 2024~року NIST опублікував перші стандарти постквантової
криптографії: FIPS~203 (ML-KEM)~\cite{fips203} та FIPS~204
(ML-DSA)~\cite{fips204}. ML-KEM базується на задачі навчання з
помилками на модульних решітках (Module-LWE), для якої не відомо
ефективних квантових алгоритмів. Гібридні підходи (класика~+ PQ)
впроваджуються в TLS~1.3~\cite{stebila2024}, Signal
(PQXDH)~\cite{brendel2020} та WireGuard~\cite{hulsing2021}.

\subsection{Обмеження існуючих рішень}

Класичний OPAQUE у поточній специфікації IETF~\cite{ietf-opaque} не
містить механізмів захисту від квантових атак: ключовий матеріал
класичного 4DH може бути відновлений алгоритмом Шора. Інші
PAKE-протоколи (SRP, SPAKE2) є так само вразливими. Чисто постквантові
PAKE перебувають на ранній стадії дослідження~\cite{ding2017}. Існуючі
гібридні рішення (TLS~1.3, Signal, WireGuard) не адресують парольну
автентифікацію. У відкритій науковій літературі відсутні
стандартизовані гібридні постквантові розширення OPAQUE.

\subsection{Мета та завдання дослідження}

\textbf{Мета:} розробка, формальна верифікація та реалізація гібридного
постквантового протоколу на основі OPAQUE, що забезпечує одночасну
стійкість до класичних та квантових атак без втрати фундаментальних
властивостей безпеки aPAKE.

\textbf{Завдання:}
\begin{enumerate}
  \item Розробити конструкцію гібридного комбінатора ключового матеріалу
        (4DH~+~ML-KEM-768).
  \item Специфікувати розширений транскрипт із постквантовими
        елементами.
  \item Провести формальну верифікацію (ProVerif~+ Tamarin) семи
        властивостей безпеки.
  \item Реалізувати протокол мовою Rust та виміряти продуктивність за
        допомогою статистичних еталонних вимірювань Criterion.
  \item Оцінити прийнятність накладних витрат для практичного
        застосування.
\end{enumerate}

\subsection{Наукова новизна та внесок}

\begin{enumerate}
  \item \textbf{Гібридна конструкція PQ-OPAQUE.} Вперше запропоновано та
        реалізовано гібридну схему розширення OPAQUE, в якій ML-KEM-768
        є \emph{обов'язковою} складовою протоколу, а не факультативним
        розширенням.
  \item \textbf{Спеціалізований комбінатор ключового матеріалу.}
        $\mathrm{PRK} = \mathrm{HKDF\text{-}Extract}(\mathrm{salt}_\tau,\;
        dh_1 \| dh_2 \| dh_3 \| dh_4 \| ss_{\mathrm{kem}})$ із
        контекстно-залежною сіллю, що містить мічений геш розширеного
        транскрипту~$\tau$.
  \item \textbf{Розширений транскрипт із PQ-елементами.} Включення
        $pk_{\mathrm{kem}}$ (1184~байти) та $ct_{\mathrm{kem}}$
        (1088~байт) у транскрипт запобігає атакам зниження рівня
        безпеки.
  \item \textbf{Повна формальна верифікація.} Сім властивостей безпеки
        верифіковано двома незалежними інструментами: ProVerif~2.05
        (5/5) та Tamarin~1.10.0 (8/8 лем, 28,08~с).
  \item \textbf{Реалізація Rust із задокументованою продуктивністю.}
        Відкрита реалізація: 4~крейти, 2729~РВК, 111~тестів (0 відмов),
        статистичні еталонні вимірювання Criterion на Apple~M1~Pro.
\end{enumerate}

% ============================================================
\section{Аналіз літературних джерел та існуючих рішень}
% ============================================================

\subsection{Протоколи парольного автентифікованого ключового обміну}

Парольний автентифікований ключовий обмін (PAKE)~--- клас протоколів,
які дозволяють двом сторонам, що поділяють спільний секрет низької
ентропії (пароль), встановити автентифікований сесійний ключ, стійкий
до офлайн-атак за словником~\cite{boyko2000}. Протоколи PAKE
поділяються на дві категорії:

\begin{itemize}
  \item \emph{Збалансовані (balanced) PAKE}~--- обидві сторони зберігають
        ідентичне парольне представлення. Приклади: EKE~\cite{bellovin1992},
        SPAKE2~\cite{ladd2023}, CPace~\cite{haase2019}.
  \item \emph{Асиметричні (augmented) PAKE, або aPAKE}~--- сервер зберігає
        лише парольний верифікатор. Приклади: SRP~\cite{wu1998},
        AuCPace~\cite{haase2019}, OPAQUE~\cite{jarecki2018}.
\end{itemize}

\begin{table}[htbp]
\caption{Порівняльна характеристика протоколів PAKE}
\label{tab:pake-comparison}
\centering\small
\begin{tabularx}{\linewidth}{L{3.5cm}CCCCC}
\toprule
Властивість & EKE & SRP & SPAKE2 & OPAQUE \\
\midrule
Тип & Balanced & Augmented & Balanced & Augmented \\
Захист пароля від сервера & Ні & Частково & Ні & Так (OPRF) \\
Стійкість до попередніх обчислень & Ні & Ні & Ні & Так \\
Пряма секретність & Так & Так & Так & Так \\
Стійкість до компрометації сервера & Ні & Часткова & Ні & Так \\
Формальне доведення безпеки & IC & Відсутнє & ROM & UC \\
Квантова стійкість & Ні & Ні & Ні & Ні \\
\bottomrule
\end{tabularx}
\end{table}

\subsection{Протокол OPAQUE: формальний опис}

Протокол OPAQUE~\cite{jarecki2018} складається з фази реєстрації та
фази автентифікації, обидві спираються на механізм забутливої
псевдовипадкової функції (OPRF). Ключовий обмін 4DH виконує чотири
операції скалярного множення: $dh_1 = sk_S \cdot PK_C$ (взаємна
автентифікація), $dh_2 = sk_S \cdot E_C$, $dh_3 = e_S \cdot PK_C$
(пряма секретність), $dh_4 = e_S \cdot E_C$ (стійкість UKS). Безпека
доведена в моделі Universal Composability (UC)~\cite{canetti2001}.

\textbf{Забутлива псевдовипадкова функція (OPRF).} Реалізація
використовує геш на групу Ristretto255~\cite{devalence2024}:
\begin{equation}
  \alpha = r \cdot H'(\mathrm{ctx} \| x),\quad
  \beta = k \cdot \alpha,\quad
  F_k(x) = r^{-1} \cdot \beta,
\end{equation}
де $r \xleftarrow{\$} \mathbb{Z}_q^*$~--- засліплюючий скаляр
клієнта, $k$~--- секретний ключ сервера.

\subsection{Постквантова криптографія та стандарти NIST 2024}

ML-KEM-768 обрано на рівні безпеки NIST Level~3 (еквівалент AES-192).
Параметри: $|pk| = 1184$~байт, $|ct| = 1088$~байт, $|ss| = 32$~байти,
безпека IND-CCA2. Конструкція:
\begin{equation}
  \mathrm{KeyGen}() \to (pk, sk);\quad
  \mathrm{Encaps}(pk) \to (ct, ss);\quad
  \mathrm{Decaps}(sk, ct) \to ss.
\end{equation}

\subsection{Нові підходи до постквантових PAKE (2023--2025)}

Основні напрямки активних досліджень:

\textbf{Решітково-орієнтовані OPRF-замінники.} Конструкції на основі
задачі Learning With Errors (LWE) пропонуються як PQ-альтернативи
Ristretto-OPRF, проте мають суттєво більший розмір сліпих елементів та
знижену ефективність~\cite{beguinet2023}.

\textbf{Ізогенні PAKE.} Атаки Castryck--Decru (2022)~\cite{castryck2023}
підірвали довіру до SIDH-базованих конструкцій. SIDH-базовані PAKE
вважаються небезпечними.

\textbf{Гібридний підхід як оптимальний.} Консенсус спільноти (IETF
CFRG, NIST PQC) схиляється до гібридизації: AND-модель безпеки
гарантує збереження захисту навіть при виявленні вразливості в одному з
компонентів~\cite{bindel2019}. На момент проведення дослідження (2025)
жоден стандарт гібридного PAKE не опублікований~--- це і є предметом
даного дослідження.

% ============================================================
\section{Постановка задачі}
% ============================================================

\subsection{Формулювання проблеми та модель загроз}

Нехай $\mathcal{C}$ (клієнт/ініціатор) та $\mathcal{S}$
(сервер/респондер)~--- дві сторони протоколу, де $\mathcal{C}$ володіє
паролем $\mathrm{pwd}$, а $\mathcal{S}$ зберігає реєстраційний запис
$\mathrm{rec}$. Позначимо через $\mathbb{G}$~--- групу Ristretto255
простого порядку~$q$ з генератором~$g$.

\textbf{Активи, що підлягають захисту:}
\begin{itemize}
  \item Пароль $\mathrm{pwd}$; статичні ключі $sk_C, sk_S$; ефемерні
        ключі $e_C, e_S, sk_{\mathrm{kem}}$;
  \item Реєстраційний запис $\mathrm{rec} = (\mathrm{envelope}, pk_C)$;
  \item Сесійний ключ $K_s$, майстер-ключ $K_m$, спільний секрет
        $ss_{\mathrm{kem}}$.
\end{itemize}

\textbf{Модель противника.} Розглядаємо три класи:
\begin{itemize}
  \item \emph{Класичний $\mathcal{A}_C$}: поліноміально обмежений,
        активний «людина посередині», може компрометувати серверне
        сховище;
  \item \emph{Квантовий $\mathcal{A}_Q$}: розв'язує DLP/ECDLP (алгоритм
        Шора), застосовує стратегію «збирай зараз~--- дешифруй пізніше»;
  \item \emph{Композитний $\mathcal{A}_{CQ}$}: активний «людина
        посередині» із квантовим обчислювальним оракулом.
\end{itemize}

\textbf{Вимоги до протоколу:}
\begin{description}
  \item[R1] (OPAQUE-властивості): таємність пароля, пряма секретність,
        взаємна автентифікація, стійкість до офлайн-перебирання;
  \item[R2] (Постквантова стійкість): захист від стратегії «збирай
        зараз~--- дешифруй пізніше»;
  \item[R3] (AND-модель): для зламу необхідно подолати \emph{обидві}
        обчислювально важкі задачі (ECDLP та Module-LWE);
  \item[R4] (Мінімальні накладні витрати): збереження 3 повідомлень та
        1,5 раундів обміну;
  \item[R5] (Реалізованість): кросплатформна промислова реалізація.
\end{description}

\subsection{Формальні визначення}

\begin{definition}[KEM]
$(\mathrm{KeyGen}, \mathrm{Encaps}, \mathrm{Decaps})$:
$\mathrm{KeyGen}() \to (pk, sk)$;
$\mathrm{Encaps}(pk) \to (ct, ss)$;
$\mathrm{Decaps}(sk, ct) \to ss$.
IND-CCA2 безпека: жоден поліноміально обмежений противник з оракулом
декапсуляції не відрізняє реальний $ss$ від випадкового.
\end{definition}

\begin{definition}[Гібридний aPAKE]
Протокол $\Pi$ між $\mathcal{C}(\mathrm{pwd})$ та
$\mathcal{S}(\mathrm{rec})$, де ключовий матеріал формується з
класичного (ECDLP-based) та постквантового (lattice KEM) джерел, а
безпека гарантується за умови складності хоча б однієї задачі.
\end{definition}

\begin{definition}[PRF-безпека HKDF]
$\mathrm{HKDF\text{-}Extract}(\mathrm{salt}, \mathrm{IKM})$ на основі
HMAC-SHA-512 є $\varepsilon$-PRF: вихід PRK обчислювально
невідрізнимий від рівномірного за перевагою не більше~$\varepsilon$.
\end{definition}

\textbf{Нотація:} $\mathbb{G}$~--- Ristretto255, $|\mathbb{G}| = q$;
малі літери~--- скаляри; великі~--- точки кривої; $\|$~---
конкатенація; $H(\cdot)$~--- SHA-512;
$\mathcal{K} = (K_s, K_m)$~--- сесійний та майстер-ключ.

% ============================================================
\section{Запропонований протокол: Hybrid PQ-OPAQUE}
% ============================================================

\subsection{Архітектура та огляд}

Гібридний PQ-OPAQUE є постквантовою гібридною модифікацією
OPAQUE~\cite{ietf-opaque}. Протокол зберігає властивість aPAKE (сервер
не отримує доступу до пароля) та додає стійкість до стратегії «збирай
зараз~--- дешифруй пізніше». Протокол має дві фази: \emph{реєстрація}
(одноразова процедура) та \emph{автентифікація} (тристороннє
повідомлення KE1--KE2--KE3).

\begin{table}[htbp]
\caption{Криптографічні примітиви Hybrid PQ-OPAQUE}
\label{tab:primitives}
\centering\small
\begin{tabularx}{\linewidth}{L{3.2cm}L{3.2cm}L{2.8cm}L{2cm}}
\toprule
Компонент & Примітив & Реалізація & Рівень безпеки \\
\midrule
Група OPRF & Ristretto255 & \texttt{libsodium} & 128 біт \\
Геш-функція & SHA-512 & \texttt{libsodium} & 256 біт \\
KDF (витяг) & HKDF-Extract (HMAC-SHA-512) & \texttt{libsodium} & 256 біт \\
KDF (розш.) & HKDF-Expand (HMAC-SHA-512) & \texttt{libsodium} & 256 біт \\
MAC & HMAC-SHA-512 & \texttt{libsodium} & 256 біт \\
KSF & Argon2id (MODERATE) & \texttt{libsodium} & Адаптивний \\
AEAD & XSalsa20-Poly1305 & \texttt{libsodium} & 256 біт \\
KEM & ML-KEM-768 (FIPS~203) & \texttt{ml-kem} & NIST Рівень~3 \\
Кл. обмін & 4DH (Ristretto255) & \texttt{libsodium} & 128 біт \\
\bottomrule
\end{tabularx}
\end{table}

\begin{table}[htbp]
\caption{Розміри повідомлень автентифікації}
\label{tab:message-sizes}
\centering\small
\begin{tabular}{lrrr}
\toprule
Повідомлення & Класичний OPAQUE & Гібридний PQ-OPAQUE & Накладні витрати \\
\midrule
KE1 & 88 байт & 1272 байт & $+1184$ ($pk_{\mathrm{kem}}$) \\
KE2 & 288 байт & 1376 байт & $+1088$ ($ct_{\mathrm{kem}}$) \\
KE3 & 64 байт & 64 байт & $0$ \\
\midrule
\textbf{Разом} & \textbf{440 байт} & \textbf{2712 байт} & \textbf{$+2272$ ($+516\%$)} \\
\bottomrule
\end{tabular}
\end{table}

\subsection{Фаза реєстрації}

\textbf{Крок~1 ($\mathcal{C} \to \mathcal{S}$).} Генерація статичної
пари $(sk_C, PK_C)$; OPRF-засліплення:
\begin{equation}
  r \xleftarrow{\$} \mathbb{Z}_q^*, \qquad
  B = r \cdot H'\!\bigl(\text{``ECLIPTIX-OPAQUE-v1/OPRF''} \| \mathrm{pwd}\bigr).
\end{equation}
$\mathrm{RegistrationRequest} = B$ (32~байти).

\textbf{Крок~2 ($\mathcal{S} \to \mathcal{C}$).} Виведення OPRF-ключа:
\begin{equation}
  k_{\mathrm{oprf}} = \mathrm{ScalarReduce}\bigl(
    \mathrm{HKDF}(\mathrm{oprf\_seed},\;
    \text{``OPRF-Key''} \| \mathrm{account\_id})\bigr).
\end{equation}
Обчислення $Z = k_{\mathrm{oprf}} \cdot B$; ключ $k_{\mathrm{oprf}}$
негайно знищується. Відповідь $= Z \| PK_S$ (64~байти).

\textbf{Крок~3 ($\mathcal{C}$).} OPRF-розсліплення
$U = r^{-1} \cdot Z$; рандомізований пароль:
\begin{equation}
  \mathrm{rwd} = \mathrm{Argon2id}\!\bigl(H(\text{``KSF''} \| U \| \mathrm{pwd}),\, \mathrm{salt}\bigr).
\end{equation}
Запечатування конверта (XSalsa20-Poly1305~+ HKDF-деривований ключ);
$\mathrm{RegistrationRecord} = \mathrm{Envelope} \| PK_C$
(168~байт).

\subsection{Фаза автентифікації: гібридний ключовий обмін}

\subsubsection{Генерація KE1 ($\mathcal{C} \to \mathcal{S}$)}

\begin{equation}
  e_C \xleftarrow{\$} \mathbb{Z}_q^*,\quad
  E_C = e_C \cdot g;\qquad
  (pk_{\mathrm{kem}}, sk_{\mathrm{kem}}) \leftarrow
    \mathrm{ML\text{-}KEM\text{-}768.KeyGen}();\qquad
  n_C \xleftarrow{\$} \{0,1\}^{192}.
\end{equation}
\begin{equation}
  \mathrm{KE1} = B \| E_C \| n_C \| pk_{\mathrm{kem}}
  \quad(32 + 32 + 24 + 1184 = 1272~\text{байти}).
\end{equation}

\subsubsection{Генерація KE2 ($\mathcal{S} \to \mathcal{C}$)}

\textbf{Чотиристоронній Діффі--Геллман (4DH):}
\begin{equation}
  dh_1 = sk_S \cdot PK_C,\quad
  dh_2 = sk_S \cdot E_C,\quad
  dh_3 = e_S \cdot PK_C,\quad
  dh_4 = e_S \cdot E_C.
\end{equation}

\textbf{Інкапсуляція ML-KEM-768:}
\begin{equation}
  (ct_{\mathrm{kem}}, ss_{\mathrm{kem}})
    \leftarrow \mathrm{ML\text{-}KEM\text{-}768.Encaps}(pk_{\mathrm{kem}}).
\end{equation}

\textbf{Розширений транскрипт} (включає PQ-елементи):
\begin{align}
  \tau &= E_C \| E_S \| n_C \| n_S \| PK_C \| PK_S \|
         \mathrm{CredResp} \| pk_{\mathrm{kem}} \| ct_{\mathrm{kem}},\\
  \tau_H &= H\!\bigl(\text{``ECLIPTIX-OPAQUE-v1/Transcript''} \| \tau\bigr).
\end{align}

\textbf{Гібридний комбінатор:}
\begin{align}
  \mathrm{combined\_ikm} &= dh_1 \| dh_2 \| dh_3 \| dh_4 \| ss_{\mathrm{kem}}
  \quad(128 + 32 = 160~\text{байт}),\\
  \mathrm{salt} &= \text{``ECLIPTIX-OPAQUE-PQ-v1/Combiner''} \| \tau_H,\\
  \mathrm{PRK} &= \mathrm{HKDF\text{-}Extract}(\mathrm{salt},
    \mathrm{combined\_ikm}).
\end{align}

\textbf{Деривація ключів:}
\begin{align}
  K_s &= \mathrm{HKDF\text{-}Expand}(\mathrm{PRK}, \text{``SessionKey''}, 64),\\
  K_m &= \mathrm{HKDF\text{-}Expand}(\mathrm{PRK}, \text{``MasterKey''}, 32),\\
  K_{\mathrm{mac}}^S &= \mathrm{HKDF\text{-}Expand}(\mathrm{PRK},
    \text{``ResponderMAC''}, 64),\\
  K_{\mathrm{mac}}^C &= \mathrm{HKDF\text{-}Expand}(\mathrm{PRK},
    \text{``InitiatorMAC''}, 64).
\end{align}
\begin{equation}
  \mathrm{KE2} = n_S \| E_S \| \mathrm{CredResp} \|
    \mathrm{HMAC}(K_{\mathrm{mac}}^S, \tau) \| ct_{\mathrm{kem}}
  \quad(1376~\text{байт}).
\end{equation}

\subsubsection{Генерація KE3 ($\mathcal{C} \to \mathcal{S}$)}

\begin{enumerate}
  \item OPRF-фіналізація, відкриття конверта, відновлення
        $(PK_S, sk_C, PK_C)$;
  \item Дзеркальний 4DH: $dh_1 = sk_C \cdot PK_S$,
        $dh_2 = e_C \cdot PK_S$, $dh_3 = sk_C \cdot E_S$,
        $dh_4 = e_C \cdot E_S$ (UKS-стійкість);
  \item Декапсуляція: $ss_{\mathrm{kem}} =
        \mathrm{ML\text{-}KEM\text{-}768.Decaps}(sk_{\mathrm{kem}},
        ct_{\mathrm{kem}})$; $sk_{\mathrm{kem}}$ негайно знищується;
  \item Ідентична деривація PRK; верифікація
        $\mathrm{HMAC}(K_{\mathrm{mac}}^S, \tau)$ рівночасовою
        функцією \texttt{crypto\_verify\_64};
  \item $\mathrm{KE3} = \mathrm{HMAC}(K_{\mathrm{mac}}^C, \tau)$
        (64~байти).
\end{enumerate}

\subsubsection{Завершення на стороні сервера}

Верифікація KE3 рівночасовою функцією порівняння. При успіху~---
пара $(K_s, K_m)$ стає доступною обом сторонам, підтверджуючи
взаємну автентифікацію. При відмові~--- всі ключові матеріали
негайно знищуються через безпечне затирання.

\subsection{Включення PQ-елементів у транскрипт}

Включення $pk_{\mathrm{kem}}$ та $ct_{\mathrm{kem}}$ у транскрипт є
\textbf{криптографічно необхідним}: воно \emph{криптографічно зв'язує}
постквантовий обмін з класичним. Без цього зв'язування противник типу
«людина посередині» може замінити $pk_{\mathrm{kem}}$ власним ключем на
рівні ML-KEM. Включення в транскрипт закриває цей вектор атаки.

% ============================================================
\section{Аналіз безпеки}
% ============================================================

\subsection{Формальний аналіз властивостей безпеки}

\begin{theorem}[Секретність паролю]
Перевага противника у визначенні пароля є нехтовно малою за умови:
(1)~забутькуватості OPRF на Ristretto255; (2)~стійкості Argon2id;
(3)~семантичної стійкості XSalsa20-Poly1305.
\end{theorem}

\begin{sketchproof}
\textbf{Гра~0 $\to$ Гра~1:} замінюємо вихід OPRF рівномірно
випадковим; обчислювальна нерозрізненність за властивістю
забутливості OPRF:
\[
  |\Pr[S_1] - \Pr[S_0]| \leq \mathrm{Adv}_{\mathrm{OPRF}}^{\mathrm{obliv}}
  \leq \mathrm{Adv}_{\mathrm{Ristretto255}}^{\mathrm{CDH}}.
\]
\textbf{Гра~1 $\to$ Гра~2:} замінюємо вихід Argon2id випадковим
ключем:
\[
  |\Pr[S_2] - \Pr[S_1]| \leq \mathrm{Adv}_{\mathrm{Argon2id}}^{\mathrm{PRF}}.
\]
\textbf{Гра~2 $\to$ Гра~3:} замінюємо конверт шифруванням
випадкового рядка (IND-CPA XSalsa20-Poly1305):
\[
  |\Pr[S_3] - \Pr[S_2]| \leq \mathrm{Adv}_{\mathrm{XSalsa20}}^{\mathrm{IND\text{-}CPA}}.
\]
У Грі~3: $\mathrm{Adv}^{\mathrm{pw}} = 1/|\mathcal{D}|$. Загальна
перевага:
\[
  \mathrm{Adv}^{\mathrm{pw}} \leq
  \mathrm{Adv}_{\mathrm{CDH}}^{\mathrm{Ristretto255}} +
  \mathrm{Adv}_{\mathrm{Argon2id}}^{\mathrm{PRF}} +
  \mathrm{Adv}_{\mathrm{XSalsa20}}^{\mathrm{IND\text{-}CPA}} +
  \frac{1}{|\mathcal{D}|}. \qquad \blacksquare
\]
\end{sketchproof}

\begin{theorem}[Пряма секретність]
Компрометація довготривалих ключів $sk_C, sk_S$ після завершення
сесії не дозволяє відновити $K_s$, за умови складності CDH та
IND-CCA безпеки ML-KEM-768.
\end{theorem}

\begin{sketchproof}
Після компрометації відомо $dh_1$. Але $dh_2 = sk_S \cdot E_C$
потребує ефемерного $e_C$ (знищено); $dh_3 = e_S \cdot PK_C$~---
ефемерного $e_S$ (знищено). Аналогічно, $ss_{\mathrm{kem}}$
захищений IND-CCA ML-KEM-768: $sk_{\mathrm{kem}}$ знищено після
декапсуляції.
\[
  \mathrm{Adv}^{\mathrm{FS}} \leq
  \mathrm{Adv}_{\mathrm{CDH}} +
  \mathrm{Adv}_{\mathrm{ML\text{-}KEM}}^{\mathrm{IND\text{-}CCA}}.
  \qquad \blacksquare
\]
\end{sketchproof}

\begin{theorem}[Взаємна автентифікація]
Ймовірність підробки MAC є нехтовно малою за умови PRF-стійкості
HMAC-SHA-512:
\[
  \mathrm{Adv}^{\mathrm{auth}} \leq
  \mathrm{Adv}_{\mathrm{HMAC}}^{\mathrm{PRF}} + q_s / 2^{512}.
  \qquad \blacksquare
\]
\end{theorem}

\begin{theorem}[Гібридна безпека, AND-модель]
Якщо хоча б одна з умов виконується: (a)~$dh_1 \| dh_2 \| dh_3$
псевдовипадкові (CDH); або (b)~$ss_{\mathrm{kem}}$ псевдовипадковий
(ML-KEM IND-CCA)~--- тоді PRK є псевдовипадковим.
\end{theorem}

\begin{sketchproof}
За dual-PRF властивістю HKDF-Extract~\cite{krawczyk2010}: якщо IKM є
псевдовипадковим (навіть частково), PRK є псевдовипадковим незалежно
від солі. Два випадки дають:
\[
  \mathrm{Adv}^{\mathrm{hybrid}} \leq
  \min\!\bigl(\mathrm{Adv}_{\mathrm{CDH}}^{\mathrm{Ristretto255}},\;
  \mathrm{Adv}_{\mathrm{ML\text{-}KEM}}^{\mathrm{IND\text{-}CCA}}\bigr)
  + \mathrm{Adv}_{\mathrm{HMAC}}^{\mathrm{dual\text{-}PRF}}.
  \qquad \blacksquare
\]
\end{sketchproof}

\subsection{Стійкість до відомих атак}

\textbf{Атаки офлайн-перебирання за словником.} Захист на трьох
незалежних рівнях: (1)~OPRF~--- без серверного ключа $k_{\mathrm{oprf}}$
противник не може обчислити $\mathrm{rwd}$; (2)~Argon2id із
параметрами MODERATE (64~МБ, 3~ітерації)~---
$C_{\mathrm{offline}} \geq |\mathcal{D}| \cdot 500$~мс;
(3)~XSalsa20-Poly1305~--- верифікація кандидата потребує успішного
автентифікованого розшифрування.

\textbf{Атака «людина посередині».} Транскрипт $\tau$ включає
\emph{всі} відкриті елементи, зокрема $pk_{\mathrm{kem}}$ та
$ct_{\mathrm{kem}}$. Модифікація будь-якого поля призводить до
незбіжності MAC.

\textbf{Атака повторного відтворення.} Ефемерні одноразові числа
(192~біти) та ефемерні ключі генеруються заново для кожної сесії.
Ймовірність колізії нонсів: $2^{-192}$.

\textbf{Атака «збирай зараз~--- дешифруй пізніше».} Навіть при
відновленні $dh_1,\ldots,dh_4$ алгоритмом Шора, секрет
$ss_{\mathrm{kem}}$ залишається захищеним ML-KEM-768
(Module-LWE, еквівалент AES-192).

\textbf{Атака на компрометований ключ (KCI).} Компрометація $sk_S$
не надає доступу до $dh_2 = sk_S \cdot E_C$ після закінчення сесії
(залежить від знищеного $e_C$) та до $ss_{\mathrm{kem}}$ (залежить
від знищеного $sk_{\mathrm{kem}}$).

\textbf{Часові бічні канали.} Усі порівняння MAC виконуються
рівночасовою функцією \texttt{crypto\_verify\_64}. Скалярне множення
на Ristretto255 та NTT в ML-KEM реалізовано в рівночасовий спосіб.

\subsection{Формальна верифікація}

\subsubsection{Огляд верифікації}

Протокол верифіковано двома незалежними символьними інструментами в
моделі Dolev--Yao: ProVerif~2.05 та Tamarin~Prover~1.10.0. Верифікація
охопила сім властивостей безпеки~\cite{cremers2012,basin2018}.

\begin{table}[htbp]
\caption{Зведені результати формальної верифікації}
\label{tab:verification-summary}
\centering\small
\begin{tabular}{cL{4cm}lll}
\toprule
\# & Властивість & ProVerif & Tamarin & Rust-тести \\
\midrule
P1 & Секретність сесійного ключа & \textbf{true} & \textbf{verified} (27 кр.) & 4 тести \\
P2 & Секретність пароля & \textbf{true} & \textbf{verified} (3 кр.) & 7 тестів \\
P3 & Класична пряма секретність & --- & \textbf{verified} (119 кр.) & 3 тести \\
P4 & Постквантова пряма секретність & --- & (через P6) & 3 тести \\
P5a & Автентифікація ініціатора & \textbf{true} & \textbf{verified} (32 кр.) & 8 тестів \\
P5b & Автентифікація респондера & \textbf{true} & \textbf{verified} (17 кр.) & --- \\
P5c & Ін'єктивна взаємна авт. & \textbf{true} & (implied) & --- \\
P6 & AND-модель гібридної безпеки & --- & \textbf{verified} (29 кр.) & 4 тести \\
P7 & Стійкість до словн. перебирання & --- & \textbf{verified} (4 кр.) & 6 тестів \\
--- & Коректність завершення & --- & \textbf{verified} (16 кр.) & 5 тестів \\
\midrule
\textbf{Разом} & & \textbf{5/5} & \textbf{8/8} & \textbf{111/111} \\
\bottomrule
\end{tabular}
\end{table}

\subsubsection{ProVerif 2.05}

Інструмент: ProVerif~2.05 (OCaml~5.4.0), macOS Darwin~25.2.0.
Моделі: \texttt{formal/hybrid\_pq\_opaque.pv} (секретність),
\texttt{formal/hybrid\_pq\_opaque\_auth.pv} (автентифікація).
P5c (ін'єктивна взаємна автентифікація):

\begin{lstlisting}
query pkC: point, pkS: point, sk: key;
  inj-event(ServerCompletesAuth(pkS, pkC, sk))
  ==> inj-event(ClientCompletesAuth(pkC, pkS, sk)).
RESULT: true
\end{lstlisting}

\subsubsection{Tamarin Prover 1.10.0}

Інструмент: Tamarin~1.10.0 (Maude~3.4).
Модель: \texttt{formal/hybrid\_pq\_opaque\_verified.spthy}.
Час верифікації: \textbf{28,08~секунди} (8~лем).

\begin{table}[htbp]
\caption{Леми Tamarin}
\label{tab:tamarin-lemmas}
\centering\small
\begin{tabularx}{\linewidth}{L{4.5cm}rX}
\toprule
Лема & Кроки & Ключова властивість \\
\midrule
\texttt{session\_key\_secrecy} (P1) & 27 & \texttt{!KU(\textasciitilde{}ek)} fails без \texttt{CorruptC} \\
\texttt{password\_secrecy} (P2) & 3 & \texttt{\textasciitilde{}pwd} не потрапляє до \texttt{Out()} \\
\texttt{forward\_secrecy} (P3) & 119 & Постсесійна компр. LTK, ефемерні залишаються свіжими \\
\texttt{auth\_initiator} (P5a) & 32 & MAC верифікація вимагає PRK; PRK~--- з DH+KEM \\
\texttt{auth\_responder} (P5b) & 17 & KE3 MAC~--- лише з правила KE3 \\
\texttt{and\_model} (P6) & 29 & Без ефемерних ключів неможливо обчислити 4DH і KEM~ss \\
\texttt{dictionary\_resistance} (P7) & 4 & Компр. БД розкриває конверт, але не пароль \\
\texttt{completion} (коректність) & 16 & Існує трейс з успішним завершенням \\
\bottomrule
\end{tabularx}
\end{table}

Лема P6 (AND-модель):
\begin{lstlisting}
All C S sk #i. SKC(C, S, sk) @ #i
  & not(Ex #j. RevEph(C) @ #j)
  & not(Ex #j. RevEph(S) @ #j)
  ==> not(Ex #j. K(sk) @ #j)
Result: verified (29 steps)
\end{lstlisting}

Лема допускає компрометацію LTK, але забороняє розкриття ефемерних
ключів. Навіть знаючи $sk_A$ та $sk_R$, противник не може обчислити
\texttt{combine(dh\_secret, kem\_ss)} без ефемерних \texttt{\textasciitilde{}ek} та
\texttt{\textasciitilde{}ksk}. Це доводить AND-модель: необхідно подолати
\emph{обидва} шари одночасно.

% ============================================================
\section{Реалізація та експериментальна оцінка}
% ============================================================

\subsection{Архітектура Rust-реалізації}

Протокол реалізовано мовою \textbf{Rust~1.93.1} у вигляді
багатокрейтного простору проєктів. Вибір Rust обумовлений:
гарантіями безпеки пам'яті на рівні системи типів (відсутність
нульових покажчиків, відсутність використання після звільнення,
інспектор запозичень); підтримкою \texttt{\#![forbid(unsafe\_code)]}
у крейтах ядра; трейтом \texttt{zeroize} для безпечного затирання
ключів у пам'яті.

\begin{table}[htbp]
\caption{Метрики вихідного коду простору проєктів}
\label{tab:code-metrics}
\centering\small
\begin{tabularx}{\linewidth}{L{2.5cm}L{4cm}rL{3.5cm}}
\toprule
Крейт & Призначення & РВК (src) & Залежності \\
\midrule
\texttt{opaque-core} & Криптографічні примітиви & 1206 & libsodium, ml-kem \\
\texttt{opaque-agent} & Ініціатор (клієнт) & 491 & opaque-core \\
\texttt{opaque-relay} & Респондер (сервер) & 416 & opaque-core \\
\texttt{opaque-ffi} & Прошарок FFI & 616 & opaque-agent, opaque-relay \\
\midrule
\textbf{Разом (виробн. код)} & & \textbf{2729} & \\
Тести & & 2700 & proptest, criterion \\
Еталонні вимір. & & 766 & criterion \\
Формальні моделі & & 1806 & Tamarin, ProVerif \\
\bottomrule
\end{tabularx}
\end{table}

\subsection{Тестова база та методологія}

Набір тестів будується за принципом відповідності тестових модулів
властивостям безпеки, встановленим у Розділі~3. Тестування
поділяється на три методологічні класи:
\begin{itemize}
  \item \emph{Детерміністичні тести}~--- конкретні протокольні
        сценарії з фіксованими або псевдовипадковими вхідними даними;
  \item \emph{Інтеграційні тести}~--- повний наскрізний цикл
        реєстрація~$\to$ автентифікація;
  \item \emph{Тести на основі властивостей} (бібліотека
        \texttt{proptest})~--- довільні вхідні дані за визначеними
        стратегіями.
\end{itemize}

\begin{table}[htbp]
\caption{Розподіл тестів за властивостями безпеки}
\label{tab:test-distribution}
\centering\small
\begin{tabularx}{\linewidth}{L{4.5cm}L{4cm}rL{2.5cm}}
\toprule
Модуль & Властивість & К-ть & Клас \\
\midrule
\texttt{p1\_session\_key\_secrecy} & P1: таємність сес. ключа & 4 & Детерміністичний \\
\texttt{p2\_password\_secrecy} & P2: таємність пароля & 7 & Детерміністичний \\
\texttt{p3\_classical\_forward\_secrecy} & P3: кл. пряма секретність & 3 & Детерміністичний \\
\texttt{p4\_pq\_forward\_secrecy} & P4: PQ пряма секретність & 3 & Детерміністичний \\
\texttt{p5\_mutual\_authentication} & P5: взаємна авт. & 8 & Детерміністичний \\
\texttt{p6\_and\_model\_hybrid\_security} & P6: AND-модель & 4 & Детерміністичний \\
\texttt{p7\_offline\_dictionary\_resistance} & P7: стійкість до словника & 6 & Детерміністичний \\
\texttt{integration} & Наскрізний цикл & 7 & Інтеграційний \\
\texttt{security\_proptest} & P1, P2, P5 (довільні) & 5 & На основі власт. \\
\texttt{crypto\_tests} & Криптографічні примітиви & 28 & Модульний \\
\texttt{envelope\_tests} & Конверт & 6 & Модульний \\
\texttt{oprf\_tests} & OPRF & 12 & Модульний \\
\texttt{pq\_kem\_tests} & ML-KEM-768 & 8 & Модульний \\
\texttt{protocol\_tests} & Серіалізація & 11 & Модульний \\
\texttt{server\_tests} & Рівень респондера & 12 & Модульний \\
\midrule
\textbf{Разом} & & \textbf{111} & \textbf{0 відмов} \\
\bottomrule
\end{tabularx}
\end{table}

\begin{lstlisting}
test result: ok. 111 passed; 0 failed; 0 ignored
(Apple M1 Pro, Rust 1.93.1, режим діагностичного складання)
\end{lstlisting}

\subsection{Методологія еталонних вимірювань}

\textbf{Апаратна платформа:} Apple~M1~Pro (8~ВП + 2~ЕЕ ядра,
16~ГБ LPDDR5, 3,228~ГГц), macOS~26.2, Rust~1.93.1.

\textbf{Фреймворк:} Criterion~0.5.1~\cite{heyer2023} застосовує
метод бутстрапу Welch: $N_s = 100$ незалежних вибірок, кожна з яких
є вибірковим середнім $\overline{t}_i$ по $n_i$~ітерацій; ітерації
підбираються автоматично до відносної похибки $\epsilon < 0{,}01$.
Вибіркове середнє $\hat{\mu}$ та 95\%-й довірчий інтервал (ДІ)
отримуються методом бутстрапу з 100\,000~перевибірок. Фаза
прогріву~--- 3~с.

\textbf{Умови:} \texttt{--release}, \texttt{opt-level~=~3},
\texttt{lto~=~false}; детермінований однопотоковий запуск.

\subsection{Еталонні вимірювання криптографічних примітивів}

\begin{table}[htbp]
\caption{Еталонні вимірювання криптографічних примітивів}
\label{tab:micro-bench}
\centering\small
\begin{tabular}{llrr}
\toprule
Примітив & Операція & $\hat{\mu}$ & 95\% ДІ \\
\midrule
Ristretto255 & Генерація ключової пари & 16,74~мкс & [16,03; 17,84] \\
Ristretto255 & Одиничне DH & 46,41~мкс & [45,94; 46,98] \\
Ristretto255 & 3DH (еталон) & 119,04~мкс & [117,10; 122,14] \\
Ristretto255 & \textbf{4DH (протокол)} & $\approx\mathbf{164}$~\textbf{мкс} & [$\approx$162; $\approx$167] \\
ML-KEM-768 & Генерація ключової пари & 41,05~мкс & [40,69; 41,68] \\
ML-KEM-768 & Інкапсуляція & 33,62~мкс & [32,89; 34,95] \\
ML-KEM-768 & Декапсуляція & 38,55~мкс & [37,65; 40,11] \\
ML-KEM-768 & \textbf{Повний раунд} & \textbf{122,01~мкс} & [119,32; 126,96] \\
OPRF & Засліплення & 56,16~мкс & [54,37; 59,36] \\
OPRF & Обчислення сервера & 47,59~мкс & [44,97; 52,45] \\
OPRF & Фіналізація & 65,68~мкс & [64,17; 67,89] \\
HKDF & Витяг & 1,32~мкс & [1,312; 1,319] \\
HKDF & Розширення (64~байти) & 1,21~мкс & [1,181; 1,245] \\
HMAC-SHA-512 & 256~байт & 1,61~мкс & [1,596; 1,641] \\
XSalsa20-Poly1305 & Шифрування 96~байт & 440,79~нс & [430,20; 459,46] \\
XSalsa20-Poly1305 & Розшифрування 96~байт & 613,42~нс & [594,15; 636,12] \\
Argon2id & \textbf{MODERATE} & \textbf{510,86~мс} & [490,31; 533,80] \\
Гібридний комбінатор & HKDF-Extract (160~байт) & 1,48~мкс & [1,453; 1,508] \\
\bottomrule
\end{tabular}
\end{table}

\textbf{Аналітичні спостереження.}
\begin{enumerate}
  \item \emph{Порівнянна вартість класичного та PQ шарів.} Повний
        раунд ML-KEM-768 ($\hat{\mu} = 122$~мкс) та 4DH
        ($\hat{\mu} \approx 164$~мкс) мають близькі порядки
        величин. Відношення PQ до класичних накладних витрат~---
        лише $\times 1{,}06$, що не змінює асимптотичний клас
        складності протоколу.
  \item \emph{Абсолютне домінування KSF.} Argon2id
        ($\hat{\mu} = 510{,}86$~мс) перевищує повний раунд
        ML-KEM-768 у $4\,175$~разів. Це свідоме проєктне рішення.
  \item \emph{Практична безвартісність гібридного комбінатора.}
        \texttt{combine\_key\_material} ($\hat{\mu} = 1{,}48$~мкс)
        вносить менше 0,001\% у загальний наскрізний час.
\end{enumerate}

\subsection{Еталонні вимірювання рівня протоколу}

\begin{table}[htbp]
\caption{Еталонні вимірювання протокольних функцій}
\label{tab:protocol-bench}
\centering\small
\begin{tabularx}{\linewidth}{L{2cm}L{3cm}rrX}
\toprule
Фаза & Функція & $\hat{\mu}$ & 95\% ДІ & Домінуючий компонент \\
\midrule
Реєстрація & \texttt{create\_request} & 72,46~мкс & [72,15; 72,77] & Засліплення OPRF \\
Реєстрація & \texttt{create\_response} & 45,92~мкс & [45,72; 46,16] & Обчислення OPRF \\
Реєстрація & \texttt{finalize} & \textbf{479,58~мс} & [472,11; 487,75] & \textbf{Argon2id} \\
Автентиф. & \texttt{generate\_ke1} & 118,83~мкс & [116,59; 121,76] & OPRF~+ ML-KEM keygen \\
Автентиф. & \texttt{generate\_ke2} & 297,03~мкс & [294,81; 299,69] & 4DH~+ ML-KEM encaps \\
Автентиф. & \texttt{generate\_ke3} & \textbf{492,34~мс} & [478,95; 509,96] & \textbf{Argon2id}~+ ML-KEM decaps \\
Автентиф. & \texttt{responder\_finish} & 1,24~мкс & [1,174; 1,323] & Перевірка MAC \\
\midrule
Повний & \texttt{authentication\_e2e} & \textbf{481,43~мс} & [471,73; 495,25] & \textbf{Argon2id} (99,8\%) \\
\bottomrule
\end{tabularx}
\end{table}

\textbf{Аналіз \texttt{generate\_ke2} = 297~мкс.} Час виконання:
\begin{equation}
  4\mathrm{DH} \approx 164 + \mathrm{encaps} \approx 34 +
  \mathrm{HKDF\text{-}Expand} \approx 6 + \mathrm{MAC} \approx 2 +
  \delta_{\mathrm{op}} \approx 91~\text{мкс}.
\end{equation}
Четвертий добуток $dh_4 = e_S \cdot E_C$ забезпечує UKS-стійкість і
пояснює зростання часу порівняно з 3DH приблизно на 41~мкс.

\subsection{Аналіз пропускної здатності респондера}

\begin{table}[htbp]
\caption{Пропускна здатність вузла-респондера}
\label{tab:throughput}
\centering\small
\begin{tabular}{lrl}
\toprule
Сценарій & $\hat{\mu}$ & Пропускна здатність \\
\midrule
\texttt{ke2\_and\_finish} (без Argon2id) & 760,82~нс & \textbf{1,31~млн авт./с} \\
\texttt{ke2\_only} (без Argon2id) & 254,26~мкс & 3,93~тис. авт./с \\
\bottomrule
\end{tabular}
\end{table}

Вузол-респондер не виконує Argon2id, тому теоретична пропускна
здатність на одному ядрі Apple~M1~Pro становить \textbf{1,31~млн авт./с}.
У реальному розгортанні обмеження визначатиметься мережевою
латентністю, а не обчислювальними ресурсами процесора.

\subsection{Кількісна оцінка накладних витрат PQ-компонента}

\begin{table}[htbp]
\caption{Накладні витрати ML-KEM-768 по фазах AKE (без Argon2id)}
\label{tab:pq-overhead}
\centering\small
\resizebox{\linewidth}{!}{%
\begin{tabular}{lrrrrl}
\toprule
Фаза AKE & Класична (4DH) & Гібридна (4DH+KEM) & Абс. надлишок & Відн. надлишок & PQ операція \\
\midrule
KE1 (ініціатор) & 73,2~мкс & 114,5~мкс & \textbf{$+41{,}3$~мкс} & $+56{,}4\%$ & Генерація ключів KEM \\
KE2 (респондер, без KSF) & 219,0~мкс & 239,5~мкс & \textbf{$+20{,}4$~мкс} & $+9{,}3\%$ & Інкапсуляція~+ комбінатор \\
KE3 (ініціатор, без KSF) & 234,0~мкс & 267,6~мкс & \textbf{$+33{,}2$~мкс} & $+14{,}2\%$ & Декапсуляція~+ комбінатор \\
\midrule
\textbf{Повний AKE (без KSF)} & \textbf{528,2~мкс} & \textbf{687,5~мкс} & \textbf{$+159{,}3$~мкс} & \textbf{$+30{,}1\%$} & ГКП + Інк.~+ Дек. \\
\bottomrule
\end{tabular}%
}
\end{table}

\textbf{Розкладання постквантових накладних витрат:}
\begin{equation}
  \Delta_{PQ} \approx
  \underbrace{41{,}4}_{\text{ген. кл.}} +
  \underbrace{33{,}8}_{\text{інкапс.}} +
  \underbrace{39{,}2}_{\text{декапс.}} +
  \underbrace{2 \times 1{,}48}_{\text{комб.}} +
  \underbrace{\delta_{\mathrm{алок.}}}_{\approx 42}
  \approx 119{,}3 + 42 = 161~\text{мкс},
\end{equation}
де $\delta_{\mathrm{алок.}}$~--- витрати алокації буферів
ML-KEM-768: $|pk_{\mathrm{kem}}| \approx 1{,}2$~КБ,
$|sk_{\mathrm{kem}}| \approx 2{,}4$~КБ,
$|ct_{\mathrm{kem}}| \approx 1{,}1$~КБ. Порівняно з Argon2id
(510,86~мс), PQ-надлишок складає $0{,}031\%$ реального наскрізного
часу.

\subsection{Порівняльний аналіз гібридних протоколів}

\begin{table}[htbp]
\caption{Порівняння гібридних постквантових протоколів}
\label{tab:protocol-comparison}
\centering\small
\resizebox{\linewidth}{!}{%
\begin{tabular}{lllll}
\toprule
Властивість & TLS~1.3 Hybrid & Signal PQXDH & WireGuard~PQ & \textbf{Hybrid PQ-OPAQUE} \\
\midrule
Класична основа & ECDH (X25519) & X3DH (X25519) & Noise IK (X25519) & \textbf{4DH (Ristretto255)} \\
PQ компонент & ML-KEM-768 & ML-KEM-768 & ML-KEM-768 & \textbf{ML-KEM-768} \\
AND-модель & Так & Так & Так & \textbf{Так} \\
Тип автентифікації & PKI-серт. & Пакет поперед. кл. & Статичні ключі & \textbf{PAKE (OPRF~+~пароль)} \\
Захист від компр. сервера & Ні & Ні & Ні & \textbf{Так (aPAKE)} \\
Захист від офл. перебирання & Ні & Ні & Ні & \textbf{Так (Argon2id)} \\
Пряма секретність (класична) & Так & Так & Ні & \textbf{Так} \\
PQ пряма секретність & Так & Так & Ні & \textbf{Так} \\
Формальна верифікація & TLS~1.3 & --- & --- & \textbf{ProVerif~+ Tamarin} \\
Обсяг рукостискання & $\approx$3200~байт & $\approx$2400~байт & $\approx$2800~байт & \textbf{2712~байт} \\
Кількість раундів & 1~RTT & 0~RTT & 1~RTT & \textbf{1,5~RTT} \\
\bottomrule
\end{tabular}%
}
\end{table}

% ============================================================
\section{Обговорення}
% ============================================================

\subsection{Відповідність вимогам}

\begin{table}[htbp]
\caption{Відповідність протоколу встановленим вимогам}
\label{tab:requirements}
\centering\small
\begin{tabularx}{\linewidth}{cL{2cm}cX}
\toprule
Вимога & Опис & Статус & Обґрунтування \\
\midrule
R1 & OPAQUE-властивості & \checkmark & aPAKE, OPRF, Argon2id, конверт залишаються незмінними; 111/111 тестів \\
R2 & PQ стійкість & \checkmark & ML-KEM-768 IND-CCA2 (FIPS~203); лема \texttt{and\_model} Tamarin \\
R3 & AND-модель & \checkmark & $\mathrm{HKDF}(dh_{1..4} \| ss_{\mathrm{kem}})$; Теорема~4~+ Tamarin P6 \\
R4 & Мін. накл. витрати & \checkmark & $+159{,}3$~мкс CPU (без KSF); $+2272$~байт; 0~доп. раундів; $0{,}031\%$ \\
R5 & Реалізованість & \checkmark & 2729~РВК Rust, 4~крейти, 111~тестів, Criterion \\
\bottomrule
\end{tabularx}
\end{table}

\subsection{Значення формальної верифікації}

Повна формальна верифікація (8/8 лем Tamarin~+ 5/5 запитів
ProVerif) є ключовим науковим результатом. \textbf{Обґрунтування
двох інструментів:} ProVerif краще масштабується для протоколів з
необмеженим числом сесій; Tamarin потужніший для складних
властивостей зі станом (пряма секретність, AND-модель).

\textbf{Лема P6 як центральний результат.} Вона формально доводить,
що Hybrid PQ-OPAQUE забезпечує безпеку навіть при \emph{повній}
компрометації довготривалих ключів~--- якщо ефемерні збережені. Це
суттєво сильніший результат, ніж типова TLS-верифікація.

\subsection{Аналіз продуктивності}

Домінування Argon2id (510~мс = 99,8\% часу виконання) є
\textbf{свідомим архітектурним рішенням}: саме висока обчислювальна
вартість гарантує практичну непридатність офлайн-атак. PQ операції
(122~мкс для ML-KEM-768, $\approx 164$~мкс для 4DH) є статистично
нехтовним доповненням.

Пропускна здатність вузла-респондера (1,31~млн авт./с на одному
ядрі) підтверджує, що сервер не є вузьким місцем навіть при великому
масштабі.

\subsection{Поточні обмеження}

\begin{enumerate}
  \item \textbf{Відповідність IETF.} Реалізація є «OPAQUE-подібною»
        і не є повністю конформною до
        \texttt{draft-irtf-cfrg-opaque}~\cite{ietf-opaque}.
  \item \textbf{Рівень зрілості PQ-стандартів.} ML-KEM-768
        (FIPS~203)~--- відносно новий стандарт (серпень~2024).
        AND-модель захищає від можливих майбутніх вразливостей.
  \item \textbf{Символьна vs. обчислювальна модель.} Tamarin/ProVerif
        верифікують символьну модель~\cite{cremers2012}. Розрив між
        рівнями є стандартним обмеженням формальної
        верифікації~\cite{basin2018}.
\end{enumerate}

\subsection{Напрямки подальших досліджень}

\begin{enumerate}
  \item \textbf{Конформність IETF.} Адаптація до
        \texttt{draft-irtf-cfrg-opaque}.
  \item \textbf{PQ цифрові підписи.} Заміна HMAC-SHA-512 на
        ML-DSA (FIPS~204) для невідмовності.
  \item \textbf{ML-KEM-512 для пристроїв IoT.} Оцінка компромісу
        безпека/ефективність.
  \item \textbf{UC-модель.} Формальне доведення в рамках Universal
        Composability~\cite{canetti2001} для PQ-розширень.
  \item \textbf{Стандартизація IETF CFRG.} Розробка пропозиції
        щодо стандартизованого PQ-розширення OPAQUE.
\end{enumerate}

% ============================================================
\section{Висновки}
% ============================================================

У даній роботі вирішено актуальну наукову задачу: розроблено,
формально верифіковано та реалізовано гібридний постквантовий
протокол PAKE, що забезпечує одночасний захист від класичних та
квантових криптоаналітичних атак.

\textbf{Теоретичний внесок.} Запропоновано конструкцію Гібридного
PQ-OPAQUE, де ML-KEM-768 є \emph{обов'язковою} складовою протоколу.
Гібридний комбінатор
\begin{equation}
  \mathrm{PRK} = \mathrm{HKDF\text{-}Extract}\!\bigl(
    \mathrm{salt}_{\tau},\;
    dh_1 \| dh_2 \| dh_3 \| dh_4 \| ss_{\mathrm{kem}}\bigr)
\end{equation}
з розширеним транскриптом реалізує AND-модель безпеки (Теорема~4).
Четвертий добуток Діффі--Геллмана ($dh_4 = e_S \cdot E_C$)
забезпечує стійкість UKS. Фундаментальні властивості OPAQUE
збережені (Теореми~1--3).

\textbf{Верифікаційний внесок.}
\begin{itemize}
  \item \textbf{ProVerif~2.05}: 5/5 запитів підтверджено;
  \item \textbf{Tamarin~Prover~1.10.0}: 8/8 лем доведено за
        28,08~с; зокрема пряма секретність (119~кроків) та
        AND-модель (29~кроків).
\end{itemize}

\textbf{Реалізаційний та вимірювальний внесок.}
\begin{itemize}
  \item \textbf{111 тестових сценаріїв: 0 відмов};
  \item $\texttt{authentication\_e2e}$ = \textbf{481,43~мс}
        (95\%~ДІ [471,73; 495,25]); Argon2id = 510,86~мс (99,8\%);
        ML-KEM-768 = 122,01~мкс;
  \item Пропускна здатність респондера = \textbf{1,31~млн авт./с};
  \item PQ накладні витрати: \textbf{$+159{,}3$~мкс} ($+30{,}1\%$)
        без KSF; \textbf{0,031\%} від наскрізного часу;
        $+2272$~байт ($+516\%$) при незмінних 1,5~обмінах.
\end{itemize}

Розроблений протокол надає системам парольної автентифікації
можливість перейти до постквантового захисту вже сьогодні, без
збільшення кількості раундів та з нехтовно малими накладними
витратами CPU. Гібридна AND-модель гарантує збереження класичного
рівня захисту навіть у разі виявлення майбутніх вразливостей у
ML-KEM-768~--- що критично важливо в умовах перехідного
постквантового періоду.

% ============================================================
\begin{thebibliography}{43}
% ============================================================

\bibitem{bonneau2012}
Bonneau~J., Herley~C., van~Oorschot~P.\,C., Stajano~F.
The quest to replace passwords: A framework for comparative evaluation of web authentication schemes~//
Proc. IEEE S\&P~2012. --- P.~553--567.

\bibitem{florencio2007}
Florencio~D., Herley~C.
A large-scale study of web password habits~//
Proc. WWW~2007. --- P.~657--666.

\bibitem{bellare2000}
Bellare~M., Pointcheval~D., Rogaway~P.
Authenticated key exchange secure against dictionary attacks~//
EUROCRYPT~2000, LNCS~1807. --- Springer, 2000. --- P.~139--155.

\bibitem{jarecki2018}
Jarecki~S., Krawczyk~H., Xu~J.
OPAQUE: An asymmetric PAKE protocol secure against pre-computation attacks~//
EUROCRYPT~2018, LNCS~10822. --- Springer, 2018. --- P.~456--486.

\bibitem{ietf-opaque}
Bourdrez~D., Krawczyk~H., Lewi~K., Wood~C.\,A.
The OPAQUE asymmetric PAKE protocol~//
IETF Internet-Draft draft-irtf-cfrg-opaque. --- 2024.

\bibitem{shor1994}
Shor~P.\,W.
Algorithms for quantum computation: Discrete logarithms and factoring~//
FOCS~1994. --- P.~124--134.

\bibitem{bernstein2017}
Bernstein~D.\,J., Lange~T.
Post-quantum cryptography~//
Nature. --- 2017. --- Vol.~549, No.~7671. --- P.~188--194.

\bibitem{roetteler2017}
Roetteler~M., Naehrig~M., Svore~K.\,M., Lauter~K.
Quantum resource estimates for computing elliptic curve discrete logarithms~//
ASIACRYPT~2017, LNCS~10625. --- Springer, 2017. --- P.~241--270.

\bibitem{mosca2018}
Mosca~M.
Cybersecurity in an era with quantum computers: Will we be ready?~//
IEEE Security \& Privacy. --- 2018. --- Vol.~16, No.~5. --- P.~38--41.

\bibitem{fips203}
National Institute of Standards and Technology.
Module-Lattice-Based Key-Encapsulation Mechanism Standard: FIPS~203. --- 2024.

\bibitem{fips204}
National Institute of Standards and Technology.
Module-Lattice-Based Digital Signature Standard: FIPS~204. --- 2024.

\bibitem{stebila2024}
Stebila~D., Fluhrer~S., Gueron~S.
Hybrid key exchange in TLS~1.3~//
IETF Internet-Draft draft-ietf-tls-hybrid-design. --- 2024.

\bibitem{brendel2020}
Brendel~J., Fiedler~R., G\"{u}nther~F., Jacobsen~C., Poettering~B.
Post-quantum security of the Signal protocol~//
SAC~2020, LNCS~12804. --- Springer, 2020. --- P.~567--591.

\bibitem{hulsing2021}
H\"{u}lsing~A., Ning~K.-C., Schwabe~P., Weber~F., Zimmermann~P.\,R.
Post-quantum WireGuard~//
IEEE S\&P~2021. --- P.~304--321.

\bibitem{ding2017}
Ding~J. et al.
Leakage of signal function with reused keys in RLWE key exchange~//
CT-RSA~2017, LNCS~10159. --- Springer, 2017. --- P.~297--314.

\bibitem{boyko2000}
Boyko~V., MacKenzie~P., Patel~S.
Provably secure password-authenticated key exchange using Diffie-Hellman~//
EUROCRYPT~2000, LNCS~1807. --- Springer, 2000. --- P.~156--171.

\bibitem{bellovin1992}
Bellovin~S.\,M., Merritt~M.
Encrypted key exchange: Password-based protocols secure against dictionary attacks~//
IEEE S\&P~1992. --- P.~72--84.

\bibitem{ladd2023}
Ladd~W., Kaduk~B., Harkins~D.
SPAKE2, a PAKE~// IETF RFC~9382. --- 2023.

\bibitem{haase2019}
Haase~B., Labrique~B.
AuCPace: Efficient verifier-based PAKE protocol tailored for the IIoT~//
IACR TCHES. --- 2019. --- Vol.~2019, No.~2. --- P.~1--48.

\bibitem{wu1998}
Wu~T.
The Secure Remote Password protocol~// NDSS~1998. --- P.~97--111.

\bibitem{devalence2024}
de~Valence~H. et al.
The ristretto255 and decaf448 groups~//
IETF Internet-Draft draft-irtf-cfrg-ristretto255-decaf448. --- 2024.

\bibitem{bindel2019}
Bindel~N. et al.
Tighter proofs of CCA security in the quantum random oracle model~//
PQCrypto~2019, LNCS~11505. --- Springer, 2019. --- P.~61--79.

\bibitem{canetti2001}
Canetti~R.
Universally composable security: A new paradigm for cryptographic protocols~//
FOCS~2001. --- P.~136--145.

\bibitem{krawczyk2010}
Krawczyk~H., Eronen~P.
HMAC-based Extract-and-Expand Key Derivation Function (HKDF)~//
IETF RFC~5869. --- 2010.

\bibitem{cremers2012}
Cremers~C., Mauw~S.
Operational Semantics and Verification of Security Protocols. --- Springer, 2012.

\bibitem{basin2018}
Basin~D., Cremers~C., Meadows~C.
Model checking security protocols~//
Handbook of Model Checking. --- Springer, 2018. --- P.~727--762.

\bibitem{biryukov2016}
Biryukov~A., Dinu~D., Khovratovich~D.
Argon2: New generation of memory-hard functions for password hashing~//
IEEE Euro S\&P~2016. --- P.~292--302.

\bibitem{heyer2023}
Heyer~C. et al.
Criterion.rs: Statistics-driven micro-benchmarking in Rust. ---
URL:~\url{https://github.com/bheisler/criterion.rs}. --- 2023.

\bibitem{beguinet2023}
Beguinet~H. et al.
Get Me out of This Lattice: CRYSTALS-Kyber and Post-Quantum Signal~//
ACM CCS~2023.

\bibitem{castryck2023}
Castryck~W., Decru~T.
An efficient key recovery attack on SIDH~//
EUROCRYPT~2023, LNCS~14008. --- Springer, 2023. --- P.~423--447.

\end{thebibliography}

\end{document}
